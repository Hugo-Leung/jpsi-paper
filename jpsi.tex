\documentclass[reprint,aps,unsortedaddress,superscriptaddress,prd,floatfix,showpacs,linenumbers]{revtex4-2}
\usepackage[utf8]{inputenc}
\usepackage{float}
\usepackage{graphicx}
\usepackage{amsmath,amsthm,amssymb}
\usepackage{verbatim}
\usepackage{url}
\usepackage{subcaption}
\usepackage[separate-uncertainty=true]{siunitx}
\usepackage{physics}
\usepackage{hyperref}
\usepackage{cleveref}
\usepackage[obeyFinal]{todonotes}
\setuptodonotes{inline}
\usepackage{multirow}
\graphicspath{{figures/}}
\DeclareSIUnit\barn{b}

\captionsetup{justification=raggedright,singlelinecheck=false}
\begin{document}

\title{Measurement of $J/\psi$ and $\psi^\prime$ production in $p+p$ and
$p+d$ interactions at \SI{120}{\GeV}}
\affiliation{Department of Engineering and Physics, Abilene Christian
University, Abilene, Texas 79699, USA}
\affiliation{Institute of Physics, Academia Sinica, Taipei, 11529, Taiwan}
\affiliation{Physics Division, Argonne National Laboratory, Lemont,
Illinois 60439, USA}
\affiliation{Fermi National Accelerator Laboratory,
Batavia, Illinois 60510, USA}
\affiliation{Lawrence Berkeley National Laboratory, Berkeley, California,
94720, USA}
\affiliation{Physics Division, Los Alamos National Laboratory, Los Alamos,
New Mexico 97545, USA}
\affiliation{Institute of Particle and Nuclear Studies, KEK, High Energy
Accelerator Research Organization, Tsukuba, Ibaraki 305-0801, Japan}
\affiliation{Department of Physics and Astronomy, Mississippi State
University, Mississippi State, Mississippi 39762, USA }
\affiliation{Department of Physics, National Kaohsiung Normal University,
Kaohsiung City 80201, Taiwan}
\affiliation{RIKEN Nishina Center for Accelerator-Based Science, Wako,
Saitama 351-0198, Japan}
\affiliation{Department of Physics and Astronomy, Rutgers, The State
University of New Jersey, Piscataway, New Jersey 08854, USA}
\affiliation{Department of Physics, Tokyo Institute of Technology, Meguro-ku,
Tokyo 152-8550, Japan}
\affiliation{Department of Physics, University of Colorado, Boulder,
Colorado 80309, USA}
\affiliation{Department of Physics, University of Illinois at Urbana-Champaign, Urbana, Illinois 61801, USA}
\affiliation{Department of Physics, University of Maryland, College Park,
Maryland 20742, USA}
\affiliation{Randall Laboratory of Physics, University of Michigan, Ann Arbor,
Michigan 48109, USA}
\affiliation{Department of Physics, Yamagata University, Yamagata City,
Yamagata 990-8560, Japan}
\affiliation{Kellogg Radiation Laboratory, California Institute of Technology,
Pasadena, California 91125, USA}

\author{C. Leung}
\affiliation{Department of Physics, University of Illinois at Urbana-Champaign,
Urbana, Illinois 61801, USA}

\author{J. Dove}
\affiliation{Department of Physics, University of Illinois at Urbana-Champaign,
Urbana, Illinois 61801, USA}

\author{B. Kerns}
\affiliation{Department of Physics, University of Illinois at Urbana-Champaign,
Urbana, Illinois 61801, USA}

\author{R. E. McClellan}
\affiliation{Department of Physics, University of Illinois at Urbana-Champaign,
Urbana, Illinois 61801, USA}

\author{S. Miyasaka}
\affiliation{Department of Physics, Tokyo Institute of Technology, Meguro-ku,
Tokyo 152-8550, Japan}

\author{D. H. Morton}
\affiliation{Randall Laboratory of Physics, University of Michigan, Ann Arbor,
Michigan 48109, USA}

\author{K. Nagai}
\affiliation{Department of Physics, Tokyo Institute of Technology, Meguro-ku,
Tokyo 152-8550, Japan}
\affiliation{Institute of Physics, Academia Sinica, Taipei, 11529, Taiwan}

\author{S. Prasad}
\affiliation{Department of Physics, University of Illinois at Urbana-Champaign,
Urbana, Illinois 61801, USA}

\author{F. Sanftl}
\affiliation{Department of Physics, Tokyo Institute of Technology, Meguro-ku,
Tokyo 152-8550, Japan}

\author{M. B. C. Scott}
\affiliation{Randall Laboratory of Physics, University of Michigan, Ann Arbor,
Michigan 48109, USA}

\author{A. S. Tadepalli}
\affiliation{Department of Physics and Astronomy, Rutgers, The State
University of New Jersey, Piscataway, New Jersey 08854, USA}

\author{C. A. Aidala}
\affiliation{Randall Laboratory of Physics, University of Michigan, Ann Arbor,
Michigan 48109, USA}
\affiliation{Physics Division, Los Alamos National Laboratory, Los Alamos,
New Mexico 97545, USA}

\author{J.  Arrington}
\affiliation{Physics Division, Argonne National Laboratory, Lemont,
Illinois 60439, USA}
\affiliation{Lawrence Berkeley National Laboratory, Berkeley, California,
94720, USA}

\author{C. Ayuso}
\affiliation{Randall Laboratory of Physics, University of Michigan, Ann Arbor,
Michigan 48109, USA}

\author{C. L. Barker}
\affiliation{Department of Engineering and Physics, Abilene Christian
University, Abilene, Texas 79699, USA}

\author{C. N. Brown}
\affiliation{Fermi National Accelerator Laboratory,
Batavia, Illinois 60510, USA}

\author{T. H. Chang}
\affiliation{Institute of Physics, Academia Sinica, Taipei, 11529, Taiwan}

\author{W. C. Chang}
\affiliation{Institute of Physics, Academia Sinica, Taipei, 11529, Taiwan}

\author{A. Chen}
\affiliation{Department of Physics, University of Illinois at Urbana-Champaign,
Urbana, Illinois 61801, USA}
\affiliation{Institute of Physics, Academia Sinica, Taipei, 11529, Taiwan}
\affiliation{Randall Laboratory of Physics, University of Michigan, Ann Arbor,
Michigan 48109, USA}

\author{D. C. Christian}
\affiliation{Fermi National Accelerator Laboratory,
Batavia, Illinois 60510, USA}

\author{B. P. Dannowitz}
\affiliation{Department of Physics, University of Illinois at Urbana-Champaign,
Urbana, Illinois 61801, USA}

\author{M. Daugherity}
\affiliation{Department of Engineering and Physics, Abilene Christian
University, Abilene, Texas 79699, USA}

\author{M. Diefenthaler}
\affiliation{Department of Physics, University of Illinois at Urbana-Champaign,
Urbana, Illinois 61801, USA}

\author{L. El Fassi}
\affiliation{Department of Physics and Astronomy, Mississippi State
University, Mississippi State, Mississippi 39762, USA }
\affiliation{Department of Physics and Astronomy, Rutgers, The State
University of New Jersey, Piscataway, New Jersey 08854, USA}

\author{D. F. Geesaman}
\affiliation{Physics Division, Argonne National Laboratory, Lemont,
Illinois 60439, USA}

\author{R. Gilman}
\affiliation{Department of Physics and Astronomy, Rutgers, The State
University of New Jersey, Piscataway, New Jersey 08854, USA}

\author{Y. Goto}
\affiliation{RIKEN Nishina Center for Accelerator-Based Science, Wako,
Saitama 351-0198, Japan}

\author{L. Guo}
\affiliation{Physics Division, Los Alamos National Laboratory, Los Alamos,
New Mexico 97545, USA}

\author{R. Guo}
\affiliation{Department of Physics, National Kaohsiung Normal University,
Kaohsiung City 80201, Taiwan}

\author{T. J. Hague}
\affiliation{Department of Engineering and Physics, Abilene Christian
University, Abilene, Texas 79699, USA}

\author{R. J. Holt}
\affiliation{Physics Division, Argonne National Laboratory, Lemont,
Illinois 60439, USA}
\affiliation{Kellogg Radiation Laboratory, California Institute of Technology,
Pasadena, California 91125, USA}

\author{D. Isenhower}
\affiliation{Department of Engineering and Physics, Abilene Christian
University, Abilene, Texas 79699, USA}

\author{E. R. Kinney}
\affiliation{Department of Physics, University of Colorado, Boulder,
Colorado 80309, USA}

\author{N. Kitts}
\affiliation{Department of Engineering and Physics, Abilene Christian
University, Abilene, Texas 79699, USA}

\author{A. Klein}
\affiliation{Physics Division, Los Alamos National Laboratory, Los Alamos,
New Mexico 97545, USA}

\author{D. W. Kleinjan}
\affiliation{Physics Division, Los Alamos National Laboratory, Los Alamos,
New Mexico 97545, USA}

\author{Y. Kudo}
\affiliation{Department of Physics, Yamagata University, Yamagata City,
Yamagata 990-8560, Japan}

\author{P.-J. Lin}
\affiliation{Department of Physics, University of Colorado, Boulder,
Colorado 80309, USA}

\author{K. Liu}
\affiliation{Physics Division, Los Alamos National Laboratory, Los Alamos,
New Mexico 97545, USA}

\author{M. X. Liu}
\affiliation{Physics Division, Los Alamos National Laboratory, Los Alamos,
New Mexico 97545, USA}

\author{W. Lorenzon}
\affiliation{Randall Laboratory of Physics, University of Michigan, Ann Arbor,
Michigan 48109, USA}

\author{N. C. R. Makins}
\affiliation{Department of Physics, University of Illinois at Urbana-Champaign,
Urbana, Illinois 61801, USA}

\author{ M. Mesquita de Medeiros}
\affiliation{Physics Division, Argonne National Laboratory, Lemont,
Illinois 60439, USA}

\author{P. L. McGaughey}
\affiliation{Physics Division, Los Alamos National Laboratory, Los Alamos,
New Mexico 97545, USA}

\author{Y. Miyachi}
\affiliation{Department of Physics, Yamagata University, Yamagata City,
Yamagata 990-8560, Japan}

\author{I. Mooney}
\affiliation{Randall Laboratory of Physics, University of Michigan, Ann Arbor,
Michigan 48109, USA}

\author{K. Nakahara}
\affiliation{Department of Physics, University of Maryland, College Park,
Maryland 20742, USA}

\author{K. Nakano}
\affiliation{Department of Physics, Tokyo Institute of Technology, Meguro-ku,
Tokyo 152-8550, Japan}
\affiliation{RIKEN Nishina Center for Accelerator-Based Science, Wako,
Saitama 351-0198, Japan}

\author{S. Nara}
\affiliation{Department of Physics, Yamagata University, Yamagata City,
Yamagata 990-8560, Japan}

\author{J. C. Peng}
\affiliation{Department of Physics, University of Illinois at Urbana-Champaign,
Urbana, Illinois 61801, USA}

\author{A. J. Puckett}
\affiliation{Physics Division, Los Alamos National Laboratory, Los Alamos,
New Mexico 97545, USA}

\author{B. J. Ramson}
\affiliation{Randall Laboratory of Physics, University of Michigan, Ann Arbor,
Michigan 48109, USA}
\affiliation{Fermi National Accelerator Laboratory,
Batavia, Illinois 60510, USA}

\author{P. E. Reimer}
\affiliation{Physics Division, Argonne National Laboratory, Lemont,
Illinois 60439, USA}

\author{J. G. Rubin}
\affiliation{Randall Laboratory of Physics, University of Michigan, Ann Arbor,
Michigan 48109, USA}
\affiliation{Physics Division, Argonne National Laboratory, Lemont,
Illinois 60439, USA}

\author{S. Sawada}
\affiliation{Institute of Particle and Nuclear Studies, KEK, High Energy
Accelerator Research Organization, Tsukuba, Ibaraki 305-0801, Japan}

\author{T. Sawada}
\affiliation{Randall Laboratory of Physics, University of Michigan, Ann Arbor,
Michigan 48109, USA}

\author{T.-A. Shibata}
\affiliation{Department of Physics, Tokyo Institute of Technology, Meguro-ku,
Tokyo 152-8550, Japan}
\affiliation{RIKEN Nishina Center for Accelerator-Based Science, Wako,
Saitama 351-0198, Japan}

\author{S. H. Shiu}
\affiliation{Institute of Physics, Academia Sinica, Taipei, 11529, Taiwan}

\author{D. Su}
\affiliation{Institute of Physics, Academia Sinica, Taipei, 11529, Taiwan}

\author{M. Teo}
\affiliation{Department of Physics, University of Illinois at Urbana-Champaign,
Urbana, Illinois 61801, USA}

\author{B. G Tice}
\affiliation{Physics Division, Argonne National Laboratory, Lemont,
Illinois 60439, USA}

\author{R. S. Towell}
\affiliation{Department of Engineering and Physics, Abilene Christian
University, Abilene, Texas 79699, USA}

\author{S. Uemura}
\affiliation{Department of Engineering and Physics, Abilene Christian
University, Abilene, Texas 79699, USA}

\author{S. Watson}
\affiliation{Department of Engineering and Physics, Abilene Christian
University, Abilene, Texas 79699, USA}

\author{S. G. Wang}
\affiliation{Institute of Physics, Academia Sinica, Taipei, 11529, Taiwan}
\affiliation{Department of Physics, National Kaohsiung Normal University,
Kaohsiung City 80201, Taiwan}

\author{A. B. Wickes}
\affiliation{Physics Division, Los Alamos National Laboratory, Los Alamos,
New Mexico 97545, USA}

\author{J. Wu}
\affiliation{Fermi National Accelerator Laboratory, Batavia, Illinois 60510, USA}

\author{Z. Xi}
\affiliation{Department of Engineering and Physics, Abilene Christian
University, Abilene, Texas 79699, USA}

\author{Z. Ye}
\affiliation{Physics Division, Argonne National Laboratory, Lemont,
Illinois 60439, USA}

\collaboration{FNAL E906/SeaQuest Collaboration}
\noaffiliation
\date{\today}

\begin{abstract}
High-mass dimuon production in $p+p$ and $p+d$ interactions has been measured
in the Fermilab SeaQuest experiment with \SI{120}{\GeV} proton beam. 
We report the $(p+d) / (p+p)$ cross section
ratios for $J/\psi$ and $\psi^\prime$ production covering the forward
rapidity region of $0.3 < x_F <0.8$. Unlike the  
$(p+d) / (p+p)$ Drell-Yan cross section ratios which are 
sensitive to the sea-quark flavor in the proton, the corresponding
ratios for charmonium production are primarily 
sensitive to the gluon contents in the proton relative to that in the
neutron. The measured charmonium cross section ratios are consistent with 
unity, in notable contrast to that of the Drell-Yan process. 
The absolute $J/\psi$ and
$\psi^\prime$ cross sections versus $x_F$ 
are also obtained and compared with NRQCD calculations. The $x_F$ dependence
of the $\psi^\prime / (J/\psi)$ cross section ratios suggests the increasing
importance of quark-antiquark annihilation process for $\psi^\prime$ 
production. 
\end{abstract}

\pacs{14.20.Dh,14.65.Bt,13.60.Hb}

\maketitle

\section{Introduction}

The SeaQuest experiment at Fermilab was designed to measure high-mass dimuons
produced in the interaction of \SI{120}{\GeV} proton beam with various targets
including liquid hydrogen, liquid deuterium, and solid nuclear 
targets~\cite{aidala2019}. Dimuons
originating from the Drell-Yan process~\cite{drell1970} as well as from the 
decay of charmonium 
states ($J/\psi$ and $\psi^\prime$) were collected simultaneously. 
Results from SeaQuest on the $(p+d) / (p+p)$ Drell-Yan cross section ratio, 
which is sensitive
to the $\bar d / \bar u$ flavor asymmetry in the proton, were reported
recently~\cite{dove2021,dove2023}. In this paper, we present results from
SeaQuest on the $J/\psi$ and $\psi^\prime$ charmonium production in $p+p$
and $p+d$ interactions with \SI{120}{\GeV} proton beam.

Unlike the Drell-Yan process which primarily involves the  
quark - antiquark annihilation via electromagnetic interaction, charmonium
production proceeds via strong interaction containing contributions from
both the quark-antiquark annihilation and the gluon-gluon
fusion processes. The simultaneous measurement of these two very different
processes provides complimentary
information on the partonic structures of the nucleon. In particular,
the $(p+d) / (p+p)$ cross section ratio for charmonium production is expected
to be sensitive to the ratio of the gluon distributions in the proton 
and deuteron.

Charge symmetry (CS) operation corresponds to  
a rotation of $180^\circ$ along the second axis of the 
isospin space. This operation interchanges up and down quarks, as well
as proton and neutron. 
%Evidence for the violation of charge symmetry at the hadronic level has been
%reported in interactions such as the $d d \to {\pi^0}~^4\mbox{He}$
%reaction~\cite{stephenson2003}
%and the $n p \to \pi^0 d$ reaction~\cite{opper2003}.
Violation of charge symmetry is predicted at the
partonic level, although there exists no experimental evidence
so far~\cite{londergan2010}. Since gluon is an iso-scalar particle, charge symmetry
requires that the gluon distributions in the proton and neutron
should be identical. A sensitive measurement of the
gluon contents in proton and neutron could provide a test of CS at
the partonic level~\cite{piller1996,zhu2008,lansberg2012}.
 
While proton-induced charmonium production is generally dominated
by the gluon-gluon fusion process~\cite{vogt1999}, the
quark-antiquark annihilation process can also contribute. 
The quark-antiquark annihilation is sensitive to the
$\bar{d} / \bar{u}$ flavor asymmetry in the proton, while the
gluon-gluon fusion is sensitive to the gluon contents in proton and
neutron. The relative importance of these two
processes depends on the beam energy as well as on the 
Feynman-$x$ ($x_F$) of the charmonium~\cite{peng1995}. Therefore, the 
beam-energy and the $x_F$ dependencies
of the $(p+d) / (p+p)$ cross section ratio for $J/\psi$ production 
could reveal the relative importance of these two processes.

The NA51 Collaboration reported a measurement of the $(p+d) / (p+p)$ cross 
section ratios for charmonium production at \SI{450}{\GeV} at a single value of
rapidity ($x_F \sim 0.$)~\cite{abreu1998}. The present measurement covers the 
kinematic range of $0.3 < x_F < 0.8$. The lower beam energy of \SI{120}{\GeV}
for the SeaQuest experiment is also expected to probe
the parton distributions at larger values of $x$, the fractional momentum
carried by the partons, than accessed by the NA51 experiment at \SI{450}{\GeV}.

%The organization of this paper is as follows. Following this introduction,
%the SeaQuest experiment will be described in Section II.
%Section III will present the procedure for data analysis and Monte-Carlo
%simulation. Results of the $x_F$ dependencies of the $p+d$ and $p+p$
%cross sections will be presented in Section IV. 
%will be presented in Section IV. The $(p+d)/(p+p)$ $J/\Psi$ cross section
%ratios will be compared with the Drell-Yan ratios, followed by Conclusions.
 
\section{The SeaQuest Experiment}

The SeaQuest experiment was performed using a \SI{120}{\GeV} proton beam from the 
Fermilab Main Injector. The SeaQuest dimuon spectrometer was designed for
detecting high-mass dimuon pairs produced in the interaction of proton
with various targets. Details of the SeaQuest spectrometer can be found
elsewhere~\cite{aidala2019,dove2021,dove2023}. A primary proton beam containing 
up to $6 \times 10^{12}$ protons in a 4-second long beam spill every minute 
was incident upon one of three identical \SI{50.8}{\cm} long cylindrical 
stainless steel
target flasks. The targets alternated between liquid hydrogen, liquid deuterium,
and the empty flask targets. A Beam Intensity Monitor (BIM) Cherenkov counter 
was placed in the beam to record the instantaneous proton intensity for 
each \SI{1}{\ns} long RF bucket at \SI{53}{\MHz} repetition rate. 
%The BIM is essential
%for measuring the integrated beam luminosity, and to inhibit triggers
%occuring within RF buckets containing large number of protons 
%exceeding a preset threshold. 
%This significantly reduces the data-acquisition
%deadtime by not triggering on events which are too noisy to be 
%reliably reconstructed.  

\begin{figure}[tb]
\includegraphics*[width=\linewidth]{massFit/LD2_massfit}
\caption{Dimuon mass distribution for events collected 
on liquid deuterium target for the second data set.
The data points (solid circles) are
compared with a fit consisting of various components. The dashed curves are
for the $J/\psi$ and $\psi^\prime$ resonances, and the dotted curve is from the
empty target flask. The accidental background is represented as the
dot-dashed curve, and the Drell-Yan dimuons are displayed as the solid curve.}
\label{Fig:LD2_Mass}
\end{figure}

%\begin{figure}[tb]
%\includegraphics*[width=\linewidth]{LD2_massfit.eps}
%\caption{Same as Fig.~\ref{Fig:LH2_Mass}, but for data collected on 
%the liquid hydrogen target.}
%\label{Fig:LD2_Mass}
%\end{figure}

The SeaQuest spectrometer consists of two dipole magnets and 
four detector stations equipped with hodoscopes and tracking chambers.
Central to the spectrometer are two large dipole magnets. 
The first magnet downstream of the target is a solid iron magnet which
also serves as a beam-dump and a hadron absorber. The second
magnet serves as a momentum analyzing magnet. Dimuon events
were triggered based on the requirement of a triple hodoscope
coincidence with a pattern consistent with a muon pair originating
from the target. Various diagnostic triggers were also implemented.
In particular, the ``single-muon" trigger was used to evaluate the 
accidental dimuon
background. The ``random" trigger sampled the
detector response during the data-taking.
% and provided input to the
%Monte-Carlo simulation of the experiment.

%\begin{table}[tbp]

%\caption{Yields and average values of kinematic variables
%for $J/\psi$ and $\psi^\prime$ events measured in $p+p$ and $p+d$ collisions in SeaQuest.}
%  \label{tab:tabyield}
%  \begin{tabular}{cccc}
%hline
%\hline
%Reaction & Yield & $\langle x_F \rangle$ 
%& $\langle p_T \rangle$ \\ \hline
%$p+p \to J/\psi$ & xx & xx & xx\\
%$p+p \to \psi^\prime$ & xx & xx & xx\\
%$p+d \to J/\psi$ & xx & xx & xx\\
%$p+d \to \psi^\prime$ & xx & xx & xx\\
%\hline
%\hline
%\end{tabular}
%\end{table}

\section{Data Analysis}

\begin{figure}[tb]
\includegraphics*[width=\linewidth]{crossSections/xF/cs_xF_full}
\caption{The differential cross
section per nucleon $d\sigma / d x_F$ for $J/\psi$ and $\psi^\prime$
production in $p+p$ and $p+d$ interaction at \SI{120}{\GeV}.
The curves correspond to NRQCD calculation using the LDMEs obtained
in \cite{chang2023} and the nucleon PDFs from \cite{hou2021}.}
\label{fig:cs_xF}
\end{figure}

The SeaQuest data sets are separated into two parts, each containing 
roughly half of the total collected data. The first part includes data 
taken between June 2014 and July 2015, and the second part covered the 
remaining period up to July 2017.
Results on the analysis of the Drell-Yan events from the first data set 
have already been reported~\cite{dove2021,dove2023}. For the charmonium data 
reported in this paper, we have analyzed the full SeaQuest data sets. 
Since the trigger conditions and the detector configuration for the two 
data sets are not identical, the analysis was performed
separately for each data set. Results obtained from the two data sets 
are first compared to check their consistency, and then combined for 
the final results.    

\begin{figure}[tb]
\includegraphics*[width=\linewidth]{crossSections/pT/cs_pT_full}
\caption{The differential cross
$d\sigma / d p^2_T$ per nucleon for $J/\psi$ and $\psi^\prime$
production in $p+p$ and $p+d$ interaction at \SI{120}{\GeV}.
The curves correspond to fits using the Kaplan form described in the
text.}
\label{fig:cs_pT}
\end{figure}

Tracks reconstructed in the drift chambers were extrapolated
to the target region. Only dimuon events consistent with originating
from the target were selected. The target
position was then used to
refine the parameters of each muon pair. The resulting
RMS mass resolution for $J/\psi$ is $\sim$\SI{200}{\MeV}.
This resolution is dominated by the finite target length and the
multiple scattering of muons in the iron magnet.

\Cref{Fig:LD2_Mass} shows the dimuon mass spectrum for the data 
collected with the
liquid deuterium target for the second data set. A comparison with the 
dimuon mass spectrum obtained for the first data set, 
reported in~\cite{dove2023}, shows
a small but noticeable difference attributed to the different 
trigger conditions. 

To extract the yields of $J/\psi$ and
$\psi^\prime$, the dimuon mass spectrum was fitted by 
including several components. First, the
data collected with empty target flask
are used to determine the background originating from sources other than the
liquid target. Second, a Monte-Carlo (M-C) based on 
GEANT4 was performed to obtain the expected line shapes of the $J/\psi$ 
and $\psi^\prime$ resonances. It first generated the ``clean" 
dimuon events, which were then embedded by the additional hits in the detectors 
according to the data collected with the ``random" trigger. This
procedure ensures that the spectrometer responses to the background hits
are properly taken into account.
Third, dimuons from the Drell-Yan process are simulated using 
next-to-leading order
calculation with the CTEQ14 parton distributions.
%These M-C
%Drell-Yan events were generated using next-to-leading order
%calculations and the CTEQ14 parton distributions.
The embedding procedure was also applied to the Drell-Yan M-C data.
Finally, the accidental background, caused by two independent 
interactions within the same RF bucket, is simulated
by forming a random combination of data collected with the 
``single-muon" trigger, as discussed in detail in Ref.~\cite{dove2023}.
These embedded M-C events were then analyzed by applying cuts identical 
to those for the real data.

A fit to the $p+d$ dimuon data, 
allowing the normalizations of
the various components except the empty flask data to vary, is 
shown in \cref{Fig:LD2_Mass}. 
The data are well described by the combination of various
components. The adequacy of this approach is further supported by the excellent
agreement for the extracted Drell-Yan $(p+d) / (p+p)$ cross section ratio 
between this method~\cite{dove2023} and an independent intensity-extrapolation
method adopted in Ref.~\cite{dove2021}.

%Figure~\ref{Fig:LH2_Mass} shows the dimuon mass distribution for 
%events passing all analysis
%cuts for the data collected on the liquid hydrogen target. The various curves
%in Fig.~\ref{Fig:LH2_Mass} correspond to the different components 
%analogous to those in Fig.~\ref{Fig:LD2_Mass}.
%The good agreement between the data and the fit indicates the suitability of
%this approach to extract the $J/\psi$ yield. Good agreement between the first
%data set and the fit was also obtained for the first data set, as shown in
%Ref.~\cite{Dove22}. 
%Table~\ref{tab:tabyield} 
%shows the numbers of $J/\psi$ and $\psi^\prime$ events for $p+p$ and $p+d$
%extracted from the analysis of both data sets. The mean values of the 
%$x_F$ and $p_T$ kinematic variables 
%are also listed in Table~\ref{tab:tabyield}. We recall
%the following definitions of the $J/\psi$ kinematic variable:
%\begin{equation}
%x_F = \frac{P_L}{P_{max}} = \frac{2P_L}{\sqrt s (1-M^2/s)},
%\label{eq:eq1}
%\end{equation}
%where $P_L$ is the longitudinal momentum of $J/\psi$ and $P_{max}$
%is the maximum momentum for $J/\psi$ in the hadron-hadron center-of-mass
%frame. $M$ and $\sqrt s$ are the $J/\psi$ mass and the hadron-hadron total
%energy, respectively. From the $J/\psi$ four momentum $Q$, the variables
%$x_1$ and $x_2$ are obtained:
%\begin{equation}
%x_1 = \frac{P_2 \cdot Q}{P_1 \cdot P}~~~ \mbox{and}~~~ 
%x_2 = \frac{P_1 \cdot Q}{P_2 \cdot P},
%\label{eq:eq2}
%\end{equation}
%where $P_1$ and $P_2$ are the four momenta of the projectile and target hadron,
%respectively, and $P$ is the sum of $P_1$ and $P_2$.

To obtain the charmonium differential cross sections,
the data were split into bins of $x_F$ and $p_T$ and each dimuon mass 
spectrum was fitted with the procedure described earlier 
to extract the $J/\psi$ and $\psi^\prime$ yield.
We recall the following definitions of $x_F$:
\begin{equation}
x_F = \frac{2P_L}{\sqrt s (1-M^2/s)},
\label{eq:eq1}
\end{equation}
where $P_L$ is the longitudinal momentum of the charmonium.
$M$ and $\sqrt s$ are the $J/\psi$ mass and the hadron-hadron total
energy, respectively.   
The charmonium production cross section is obtained as follows
\begin{equation}
d\sigma = \frac{Y}{Acc \cdot Eff \cdot Lum}, 
\label{eq:eq2}
\end{equation}    
where $Y$ is the number of $J/\psi$ or $\psi^\prime$ events
for each $x_F$ or $p_T$ bin, $Acc$ the spectrometer acceptance, 
$Eff$ the efficiency for  analysis cuts, and $Lum$ the effective 
luminosity including the data-acquisition deadtime. 

The values of $d\sigma/ d x_F$ per target nucleon for $J/\psi$ and
$\psi^\prime$ production in $p+p$ and $p+d$ collisions are shown in 
\cref{fig:cs_xF} and listed
in \cref{tab:xfcros}. The $d\sigma/ d x_F$ differential cross sections are
obtained by using a $p_T$ distribution in the acceptance calculation
which best fits the data. The systematic uncertainties
include an overall normalization
uncertainty, common to both the $p+p$ and $p+d$ cross sections.
Other systematic errors are from the uncertainty of the fitting 
of the mass distributions and the detector efficiencies.
\todo{Add the discussion on the NRQCD calculations and the comparison
with the data. Also mention the broader xF distribution for 
$\psi^\prime$ versus $J/\psi$.}

\begin{table*}
  \caption{The differential cross sections per nucleon, $d\sigma/dx_F$
(in \unit{\nano\barn}),
for $J/\psi$ and $\psi^\prime$ production in $p+p$ and $p+d$
collisions at 120 GeV for different $x_F$ bins.
The statistical and the systematic uncertainties are also shown.}
	\begin{center}
	  \label{tab:xfcros}
  \begin{tabular}{cccc|cccc}
\hline
\multicolumn{4}{c|}{$p+p$}                                                                                                                                                                                     & \multicolumn{4}{c}{$p+d$}                                                                                                                                                                                     \\ \hline
\multicolumn{1}{c}{$\expval{x_F}_{J/\psi}$} & \multicolumn{1}{c}{$\eval{d\sigma/dx_F}_{J/\psi}$} & \multicolumn{1}{c}{$\expval{x_F}_{\psi^\prime}$} & \multicolumn{1}{c|}{$\eval{d\sigma/dx_F}_{\psi^\prime}$} & \multicolumn{1}{c}{$\expval{x_F}_{J/\psi}$} & \multicolumn{1}{c}{$\eval{d\sigma/dx_F}_{J/\psi}$} & \multicolumn{1}{c}{$\expval{x_F}_{\psi^\prime}$} & \multicolumn{1}{c}{$\eval{d\sigma/dx_F}_{\psi^\prime}$} \\ \hline
0.527                                       & $6.975\pm0.273\pm1.015$                            & 0.509                                            & $1.6258\pm0.1098\pm0.1918$                               & 0.527                                       & $7.876\pm0.261\pm0.975$                            & 0.509                                            & $1.8525\pm0.0951\pm0.1640$                              \\
0.625                                       & $2.909\pm0.116\pm0.308$                            & 0.624                                            & $0.8903\pm0.0621\pm0.0986$                               & 0.625                                       & $2.993\pm0.128\pm0.366$                            & 0.624                                            & $0.9324\pm0.0682\pm0.1130$                              \\
0.672                                       & $1.783\pm0.066\pm0.166$                            & 0.672                                            & $0.5522\pm0.0409\pm0.0567$                               & 0.672                                       & $1.868\pm0.070\pm0.183$                            & 0.672                                            & $0.6573\pm0.0421\pm0.0614$                              \\
0.733                                       & $0.921\pm0.029\pm0.097$                            & 0.733                                            & $0.3278\pm0.0212\pm0.0370$                               & 0.732                                       & $0.943\pm0.031\pm0.107$                            & 0.733                                            & $0.3149\pm0.0246\pm0.0432$                              \\
0.816                                       & $0.177\pm0.007\pm0.021$                            & 0.823                                            & $0.0642\pm0.0068\pm0.0097$                               & 0.817                                       & $0.183\pm0.007\pm0.020$                            & 0.823                                            & $0.0723\pm0.0074\pm0.0089$                              \\ \hline
\end{tabular}

  \end{center}
\end{table*}

The $p_T$ dependencies of the $J/\psi$ and $\psi^\prime$
cross sections are shown in \cref{fig:cs_pT} and listed in \cref{tab:ptcros} 
for $p+p$ and $p+d$. The $d \sigma / d p_T^2$ differential
cross sections are obtained by integrating over the $-1<x_F<1$ range
using a parametrization which best describes the data.
These $p_T$ distributions are fitted with
the parametrization $d\sigma/dp_t^2 = c (1+p_t^2/p_0^2)^{-6}$\cite{kaplan1978}
and the results of the fits are shown in \cref{fig:cs_pT}.

\begin{table*}
  \caption{The differential cross sections per nucleon, $d\sigma / dp^2_T$
 (in \unit{\nano\barn\per\GeV\squared}),
for charmonium production in $p+p$ and $p+d$
collisions at 120 GeV for different $p_T$ bins. 
The statistical and the systematic uncertainties are also shown.}
  \label{tab:ptcros}
  \begin{center}
\begin{tabular}{cccc|cccc}
\hline
\multicolumn{4}{c|}{$p+p$}                                                                                                                                                                                     & \multicolumn{4}{c}{$p+d$}                                                                                                                                                                                     \\ \hline
\multicolumn{1}{c}{$\expval{p_T}_{J/\psi}$} & \multicolumn{1}{c}{$\eval{d\sigma/dp_T}_{J/\psi}$} & \multicolumn{1}{c}{$\expval{p_T}_{\psi^\prime}$} & \multicolumn{1}{c|}{$\eval{d\sigma/dp_T}_{\psi^\prime}$} & \multicolumn{1}{c}{$\expval{p_T}_{J/\psi}$} & \multicolumn{1}{c}{$\eval{d\sigma/dp_T}_{J/\psi}$} & \multicolumn{1}{c}{$\expval{p_T}_{\psi^\prime}$} & \multicolumn{1}{c}{$\eval{d\sigma/dp_T}_{\psi^\prime}$} \\ \hline
0.194                                       & $47.48\pm2.66\pm5.51$                              & 0.194                                            & $10.81\pm0.60\pm0.92$                                    & 0.193                                       & $51.18\pm2.87\pm5.11$                              & 0.194                                            & $11.44\pm0.62\pm0.84$                                   \\
0.376                                       & $41.15\pm2.32\pm4.85$                              & 0.376                                            & $9.23\pm0.50\pm0.93$                                     & 0.376                                       & $42.45\pm2.44\pm5.28$                              & 0.377                                            & $9.67\pm0.52\pm0.86$                                    \\
0.550                                       & $30.04\pm1.43\pm3.61$                              & 0.550                                            & $7.23\pm0.33\pm0.81$                                     & 0.550                                       & $31.65\pm1.52\pm3.90$                              & 0.553                                            & $7.06\pm0.33\pm0.63$                                    \\
0.760                                       & $18.96\pm0.98\pm2.51$                              & 0.764                                            & $4.00\pm0.26\pm0.82$                                     & 0.760                                       & $18.34\pm0.99\pm3.12$                              & 0.763                                            & $3.91\pm0.28\pm0.98$                                    \\
1.095                                       & $6.43\pm0.35\pm0.81$                               & 1.107                                            & $1.17\pm0.11\pm0.35$                                     & 1.098                                       & $6.88\pm0.38\pm0.95$                               & 1.111                                            & $1.13\pm0.12\pm0.40$                                    \\ \hline
\end{tabular}

  \end{center}
\end{table*}

The $\sigma(p+d)/2\sigma(p+p)$ ratios for $J/\psi$ production are shown in 
\cref{fig:pdpp} as a function of $x_F$.
As a result of the identical target geometry
of the two liquid targets and the frequent interchange between the targets,
most of the systematic uncertainties cancel out in the 
$(p+d)/(p+p)$ cross section ratio. The remaining uncertainties in the 
SeaQuest $(p+d)/(p+p)$ $J/\psi$
and $\psi^\prime$ cross section 
ratios shown in \cref{fig:pdpp} have dominant contributions from the uncertainties
associated with the extraction of the yields $Y$.

Also shown in \cref{fig:pdpp} is the $(p+d)/(p+p)$ $J/\psi$
and $\psi^\prime$ cross section
ratios measured by the NA51 Collaboration~\cite{abreu1998} with \SI{450}{\GeV} proton 
beam. Both the NA51 and the SeaQuest
results show the cross section ratios for $J/\psi$
production are closer to unity than for 
the Drell-Yan process~\cite{dove2023}, also 
shown in \cref{fig:pdpp}. 

\begin{figure}[tb]
\includegraphics*[width=\linewidth]{crossSections/xF/pdpp_xF}
\caption{The $p+d/p+p$ per nucleon cross section ratios for $J/\psi$
versus $x_F$ from SeaQuest.
For comparison, the ratios for the Drell-Yan cross sections~\cite{dove2023} are
also shown. The solid curve is the calculation for the $J/\psi$ cross section
ratios at \SI{120}{\GeV} in the NRQCD model using the proton PDFs from NNPDF4.0.
The dashed curve represents the NLO Drell-Yan cross section ratios
calculated with the same PDFs~\cite{dove2023}.}
\label{fig:pdpp}
\end{figure}

\begin{figure}[tb]
\missingfigure{Energy dependence of the per nucleon $J/\psi$ production
cross section}
\caption{Energy dependence of the per nucleon $J/\psi$ production
cross section with proton beam. The solid curve corresponds to the NRQCD
calculation, and the dashed curve is an empirical fit to the data
as discussed in the text.}
\label{fig5}
\end{figure}

\begin{figure}[tb]
\includegraphics*[width=\linewidth]{crossSections/pT/pT_s_release}
\caption{Energy dependence of the $\langle p_T \rangle$
for $J/\psi$ production with proton beam. The curve corresponds to
an empirical fit to the data
as discussed in the text.}
\label{fig6}
\end{figure}


The difference between the Drell-Yan and the charmonium
cross section ratios clearly reflect the different underlying mechanisms in 
these two processes. The Drell-Yan process, dominated by the
$\bar u u$  annihilation subprocess, leads to the relation
\begin{align}
\sigma(p+d)_{DY}/2\sigma(p+p)_{DY} & \approx &
\frac{1}{2} (1+ \bar u_n(x_t)/\bar u_p(x_t)) \nonumber \\
 & = & \frac{1}{2} (1 + \bar d_p(x_t)/\bar u_p(x_t)),
\label{eq:DY}
\end{align}
where $\bar q_{p,n}$ denotes the $\bar q$ distribution in the proton ($p$)
and neutron ($n$), respectively. \Cref{eq:DY} clearly shows that the Drel-Yan
cross section ratios are sensitive to the sea-quark flavor asymmetry 
$\bar d(x) / \bar u(x)$. For charmonium production, the dominance of the 
gluon-gluon fusion subprocess at this beam energy implies that
\begin{align}
\sigma(p+d \to J/\psi)/2\sigma(p+p \to J/\psi) \nonumber \\
\approx \frac{1}{2} (1+ g_n(x_t)/g_p(x_t)),
\label{eq:jpsi}
\end{align}
where $g_{p,n}$ refers to the gluon distribution in the proton
and neutron, respectively. As gluon is an iso-scalar particle, one expects
$g_p(x) = g_n(x)$ and \cref{eq:jpsi} leads to identical charmonium production cross
section per nucleon for $p+p$ and $p+p$. This prediction of identical
per nucleon cross section for $p+p$ and $p+d$ will be modified once the 
contribution from $q \bar q$ annihilation is included, as discussed in the
next Section.  

%\begin{figure}[tb]
%\includegraphics*[width=\linewidth]{jpsi_fig4.pdf}
%\caption{Calculation for the $p+d \to J/\Psi$ differential cross
%section $d\sigma / d x_F$ at 120 GeV in the NLO Color-Evaporation model
%using CTEQ18NLO parton distribution functions. The dashed and dotted curve 
%corresponds to the contribution from the $q \bar q$ annihilation and
%the $g g$ fusion process, respectively. The arrow indicates the mean
%$x_F$ value for the $J/\Psi$ production in SeaQuest.}
%\label{fig4}
%\end{figure}

To compare the measured $p+d / p+p$ cross section ratios for charmonium 
production with theoretical expectation, we have performed calculation
using the Next-to-Leading order (NLO) Color Evaporation 
Model (CEM)~\cite{fritzsch1977}.
An important feature of the CEM is that it is essentially parameter-free, 
with only a single parameter that accounts for the probability of a
heavy quark-antiquark ($\bar Q Q$) pair to hadronize into a specific 
quarkonium bound state. Despite its simplicity, the CEM
gives a good description of many features of fixed-target $J/\psi$ 
cross section data with proton and pion beams~\cite{schuler1996,chang2020}, 
including their longitudinal momentum ($x_F$) distributions.

Taking charmonium production as an example, the NLO CEM first
produces a $\bar c c$ pair via various QCD hard processes at 
the NLO~\cite{mangano1993}.
For $\bar c c$ pair with an invariant mass $M_{\bar c c}$ smaller than 
the $D \bar D$ open-charm threshold, a
constant probability $F$, specific for each charmonium state,
accounts for the hadronization of $\bar c c$ pair into the colorless
charmonium state. 

\todo{
We still need to add the discussion on the results of Fig. 3 and Fig. 4
after the new figures become available.
}
%\begin{figure}[tb]
%\includegraphics*[width=\linewidth]{jpsi_fig3.pdf}
%\caption{The $p+d/p+p$ per nucleon cross section ratios for $J/\psi$ and
%$\psi^\prime$ versus $x_t$. Both the result from the SeaQuest at 120 GeV proton
%beam energy and that from the NA51 measurement~\cite{NA51} with 450
%GeV proton are shown.
%For comparison, the ratios for the Drell-Yan cross sections~\cite{Dove} are
%also shown. The solid curve is the calculation for the $J/\psi$ cross section
%ratios at 120 GeV in the NLO CEM model using the CT18NLO PDFs. The dashed
%curve represents the NLO Drell-Yan cross section ratios calculated with the
%same PDFs~\cite{Dove}.}
%\label{fig6}
%\end{figure}
\section{Conclusion}
In summary, the simultaneous measurement of the charmonium 
production and the Drell-Yan
dimuons in the SeaQuest experiment allows a direct comparison between 
the $(p+d)/(p+p)$ cross section ratios for these two distinct reactions.
While the Drell-Yan process proceeds via electromagnetic interaction sensitive
to the quarks and aniquarks, the charmonium production is a strong interaction
process sensitive to the gluon contents of the colliding nucleons. These
different characteristics between the two processes are clearly reflected
in their different $(p+d)/(p+p)$ cross section ratios. While the ratios for
the Drell-Yan process is significantly different from unity as a result of
the flavor asymmetry of the light-quark sea, the charmonium production
cross section ratios are found to be consistent with unity. This result
is in qualitative agreement with earlier measurement at higher beam energy
by the NA51 Collaboration, as well as the $\Upsilon$ production at 800 GeV
by the E866 Collaboration. A combined analysis of these data is expected
to provide a sensitive test on the assumption of identical gluon
distributions in the proton and neutron. 

\bibliography{reference}
 
\end{document}

