\documentclass[twocolumn,aps,unsortedaddress,superscriptaddress,prd,floatfix,showpacs,linenumbers]{revtex4-2}
\usepackage[utf8]{inputenc}
\usepackage{float}
\usepackage{graphicx}
\usepackage{amsmath,amsthm,amssymb}
\usepackage{verbatim}
\usepackage{subcaption}
\usepackage[separate-uncertainty=true]{siunitx}
\usepackage{physics}
\usepackage{cleveref}
\usepackage[obeyFinal]{todonotes}
\setuptodonotes{inline}
\graphicspath{{figures/}}

\begin{document}
\title{Measurement of $(p+d) / (p+p)$ cross section ratios for
$J/\Psi$ and $\Psi^\prime$ production at \SI{120}{\GeV}}
\affiliation{Department of Engineering and Physics, Abilene Christian
University, Abilene, Texas 79699, USA}
\affiliation{Institute of Physics, Academia Sinica, Taipei, 11529, Taiwan}
\affiliation{Physics Division, Argonne National Laboratory, Lemont,
Illinois 60439, USA}
\affiliation{Fermi National Accelerator Laboratory,
Batavia, Illinois 60510, USA}
\affiliation{Lawrence Berkeley National Laboratory, Berkeley, California,
94720, USA}
\affiliation{Physics Division, Los Alamos National Laboratory, Los Alamos,
New Mexico 97545, USA}
\affiliation{Institute of Particle and Nuclear Studies, KEK, High Energy
Accelerator Research Organization, Tsukuba, Ibaraki 305-0801, Japan}
\affiliation{Department of Physics and Astronomy, Mississippi State
University, Mississippi State, Mississippi 39762, USA }
\affiliation{Department of Physics, National Kaohsiung Normal University,
Kaohsiung City 80201, Taiwan}
\affiliation{RIKEN Nishina Center for Accelerator-Based Science, Wako,
Saitama 351-0198, Japan}
\affiliation{Department of Physics and Astronomy, Rutgers, The State
University of New Jersey, Piscataway, New Jersey 08854, USA}
\affiliation{Department of Physics, Tokyo Institute of Technology, Meguro-ku,
Tokyo 152-8550, Japan}
\affiliation{Department of Physics, University of Colorado, Boulder,
Colorado 80309, USA}
\affiliation{Department of Physics, University of Illinois at Urbana-Champaign, Urbana, Illinois 61801, USA}
\affiliation{Department of Physics, University of Maryland, College Park,
Maryland 20742, USA}
\affiliation{Randall Laboratory of Physics, University of Michigan, Ann Arbor,
Michigan 48109, USA}
\affiliation{Department of Physics, Yamagata University, Yamagata City,
Yamagata 990-8560, Japan}
\affiliation{Kellogg Radiation Laboratory, California Institute of Technology,
Pasadena, California 91125, USA}

\author{C. Leung}
\affiliation{Department of Physics, University of Illinois at Urbana-Champaign,
Urbana, Illinois 61801, USA}

\author{J. Dove}
\affiliation{Department of Physics, University of Illinois at Urbana-Champaign,
Urbana, Illinois 61801, USA}

\author{B. Kerns}
\affiliation{Department of Physics, University of Illinois at Urbana-Champaign,
Urbana, Illinois 61801, USA}

\author{R. E. McClellan}
\affiliation{Department of Physics, University of Illinois at Urbana-Champaign,
Urbana, Illinois 61801, USA}

\author{S. Miyasaka}
\affiliation{Department of Physics, Tokyo Institute of Technology, Meguro-ku,
Tokyo 152-8550, Japan}

\author{D. H. Morton}
\affiliation{Randall Laboratory of Physics, University of Michigan, Ann Arbor,
Michigan 48109, USA}

\author{K. Nagai}
\affiliation{Department of Physics, Tokyo Institute of Technology, Meguro-ku,
Tokyo 152-8550, Japan}
\affiliation{Institute of Physics, Academia Sinica, Taipei, 11529, Taiwan}

\author{S. Prasad}
\affiliation{Department of Physics, University of Illinois at Urbana-Champaign,
Urbana, Illinois 61801, USA}

\author{F. Sanftl}
\affiliation{Department of Physics, Tokyo Institute of Technology, Meguro-ku,
Tokyo 152-8550, Japan}

\author{M. B. C. Scott}
\affiliation{Randall Laboratory of Physics, University of Michigan, Ann Arbor,
Michigan 48109, USA}

\author{A. S. Tadepalli}
\affiliation{Department of Physics and Astronomy, Rutgers, The State
University of New Jersey, Piscataway, New Jersey 08854, USA}

\author{C. A. Aidala}
\affiliation{Randall Laboratory of Physics, University of Michigan, Ann Arbor,
Michigan 48109, USA}
\affiliation{Physics Division, Los Alamos National Laboratory, Los Alamos,
New Mexico 97545, USA}

\author{J.  Arrington}
\affiliation{Physics Division, Argonne National Laboratory, Lemont,
Illinois 60439, USA}
\affiliation{Lawrence Berkeley National Laboratory, Berkeley, California,
94720, USA}

\author{C. Ayuso}
\affiliation{Randall Laboratory of Physics, University of Michigan, Ann Arbor,
Michigan 48109, USA}

\author{C. L. Barker}
\affiliation{Department of Engineering and Physics, Abilene Christian
University, Abilene, Texas 79699, USA}

\author{C. N. Brown}
\affiliation{Fermi National Accelerator Laboratory,
Batavia, Illinois 60510, USA}

\author{T. H. Chang}
\affiliation{Institute of Physics, Academia Sinica, Taipei, 11529, Taiwan}

\author{W. C. Chang}
\affiliation{Institute of Physics, Academia Sinica, Taipei, 11529, Taiwan}

\author{A. Chen}
\affiliation{Department of Physics, University of Illinois at Urbana-Champaign,
Urbana, Illinois 61801, USA}
\affiliation{Institute of Physics, Academia Sinica, Taipei, 11529, Taiwan}
\affiliation{Randall Laboratory of Physics, University of Michigan, Ann Arbor,
Michigan 48109, USA}

\author{D. C. Christian}
\affiliation{Fermi National Accelerator Laboratory,
Batavia, Illinois 60510, USA}

\author{B. P. Dannowitz}
\affiliation{Department of Physics, University of Illinois at Urbana-Champaign,
Urbana, Illinois 61801, USA}

\author{M. Daugherity}
\affiliation{Department of Engineering and Physics, Abilene Christian
University, Abilene, Texas 79699, USA}

\author{M. Diefenthaler}
\affiliation{Department of Physics, University of Illinois at Urbana-Champaign,
Urbana, Illinois 61801, USA}

\author{L. El Fassi}
\affiliation{Department of Physics and Astronomy, Mississippi State
University, Mississippi State, Mississippi 39762, USA }
\affiliation{Department of Physics and Astronomy, Rutgers, The State
University of New Jersey, Piscataway, New Jersey 08854, USA}

\author{D. F. Geesaman}
\affiliation{Physics Division, Argonne National Laboratory, Lemont,
Illinois 60439, USA}

\author{R. Gilman}
\affiliation{Department of Physics and Astronomy, Rutgers, The State
University of New Jersey, Piscataway, New Jersey 08854, USA}

\author{Y. Goto}
\affiliation{RIKEN Nishina Center for Accelerator-Based Science, Wako,
Saitama 351-0198, Japan}

\author{L. Guo}
\affiliation{Physics Division, Los Alamos National Laboratory, Los Alamos,
New Mexico 97545, USA}

\author{R. Guo}
\affiliation{Department of Physics, National Kaohsiung Normal University,
Kaohsiung City 80201, Taiwan}

\author{T. J. Hague}
\affiliation{Department of Engineering and Physics, Abilene Christian
University, Abilene, Texas 79699, USA}

\author{R. J. Holt}
\affiliation{Physics Division, Argonne National Laboratory, Lemont,
Illinois 60439, USA}
\affiliation{Kellogg Radiation Laboratory, California Institute of Technology,
Pasadena, California 91125, USA}

\author{D. Isenhower}
\affiliation{Department of Engineering and Physics, Abilene Christian
University, Abilene, Texas 79699, USA}

\author{E. R. Kinney}
\affiliation{Department of Physics, University of Colorado, Boulder,
Colorado 80309, USA}

\author{N. Kitts}
\affiliation{Department of Engineering and Physics, Abilene Christian
University, Abilene, Texas 79699, USA}

\author{A. Klein}
\affiliation{Physics Division, Los Alamos National Laboratory, Los Alamos,
New Mexico 97545, USA}

\author{D. W. Kleinjan}
\affiliation{Physics Division, Los Alamos National Laboratory, Los Alamos,
New Mexico 97545, USA}

\author{Y. Kudo}
\affiliation{Department of Physics, Yamagata University, Yamagata City,
Yamagata 990-8560, Japan}

\author{P.-J. Lin}
\affiliation{Department of Physics, University of Colorado, Boulder,
Colorado 80309, USA}

\author{K. Liu}
\affiliation{Physics Division, Los Alamos National Laboratory, Los Alamos,
New Mexico 97545, USA}

\author{M. X. Liu}
\affiliation{Physics Division, Los Alamos National Laboratory, Los Alamos,
New Mexico 97545, USA}

\author{W. Lorenzon}
\affiliation{Randall Laboratory of Physics, University of Michigan, Ann Arbor,
Michigan 48109, USA}

\author{N. C. R. Makins}
\affiliation{Department of Physics, University of Illinois at Urbana-Champaign,
Urbana, Illinois 61801, USA}

\author{ M. Mesquita de Medeiros}
\affiliation{Physics Division, Argonne National Laboratory, Lemont,
Illinois 60439, USA}

\author{P. L. McGaughey}
\affiliation{Physics Division, Los Alamos National Laboratory, Los Alamos,
New Mexico 97545, USA}

\author{Y. Miyachi}
\affiliation{Department of Physics, Yamagata University, Yamagata City,
Yamagata 990-8560, Japan}

\author{I. Mooney}
\affiliation{Randall Laboratory of Physics, University of Michigan, Ann Arbor,
Michigan 48109, USA}

\author{K. Nakahara}
\affiliation{Department of Physics, University of Maryland, College Park,
Maryland 20742, USA}

\author{K. Nakano}
\affiliation{Department of Physics, Tokyo Institute of Technology, Meguro-ku,
Tokyo 152-8550, Japan}
\affiliation{RIKEN Nishina Center for Accelerator-Based Science, Wako,
Saitama 351-0198, Japan}

\author{S. Nara}
\affiliation{Department of Physics, Yamagata University, Yamagata City,
Yamagata 990-8560, Japan}

\author{J. C. Peng}
\affiliation{Department of Physics, University of Illinois at Urbana-Champaign,
Urbana, Illinois 61801, USA}

\author{A. J. Puckett}
\affiliation{Physics Division, Los Alamos National Laboratory, Los Alamos,
New Mexico 97545, USA}

\author{B. J. Ramson}
\affiliation{Randall Laboratory of Physics, University of Michigan, Ann Arbor,
Michigan 48109, USA}
\affiliation{Fermi National Accelerator Laboratory,
Batavia, Illinois 60510, USA}

\author{P. E. Reimer}
\affiliation{Physics Division, Argonne National Laboratory, Lemont,
Illinois 60439, USA}

\author{J. G. Rubin}
\affiliation{Randall Laboratory of Physics, University of Michigan, Ann Arbor,
Michigan 48109, USA}
\affiliation{Physics Division, Argonne National Laboratory, Lemont,
Illinois 60439, USA}

\author{S. Sawada}
\affiliation{Institute of Particle and Nuclear Studies, KEK, High Energy
Accelerator Research Organization, Tsukuba, Ibaraki 305-0801, Japan}

\author{T. Sawada}
\affiliation{Randall Laboratory of Physics, University of Michigan, Ann Arbor,
Michigan 48109, USA}

\author{T.-A. Shibata}
\affiliation{Department of Physics, Tokyo Institute of Technology, Meguro-ku,
Tokyo 152-8550, Japan}
\affiliation{RIKEN Nishina Center for Accelerator-Based Science, Wako,
Saitama 351-0198, Japan}

\author{S. H. Shiu}
\affiliation{Institute of Physics, Academia Sinica, Taipei, 11529, Taiwan}

\author{D. Su}
\affiliation{Institute of Physics, Academia Sinica, Taipei, 11529, Taiwan}

\author{M. Teo}
\affiliation{Department of Physics, University of Illinois at Urbana-Champaign,
Urbana, Illinois 61801, USA}

\author{B. G Tice}
\affiliation{Physics Division, Argonne National Laboratory, Lemont,
Illinois 60439, USA}

\author{R. S. Towell}
\affiliation{Department of Engineering and Physics, Abilene Christian
University, Abilene, Texas 79699, USA}

\author{S. Uemura}
\affiliation{Department of Engineering and Physics, Abilene Christian
University, Abilene, Texas 79699, USA}

\author{S. Watson}
\affiliation{Department of Engineering and Physics, Abilene Christian
University, Abilene, Texas 79699, USA}

\author{S. G. Wang}
\affiliation{Institute of Physics, Academia Sinica, Taipei, 11529, Taiwan}
\affiliation{Department of Physics, National Kaohsiung Normal University,
Kaohsiung City 80201, Taiwan}

\author{A. B. Wickes}
\affiliation{Physics Division, Los Alamos National Laboratory, Los Alamos,
New Mexico 97545, USA}

\author{J. Wu}
\affiliation{Fermi National Accelerator Laboratory, Batavia, Illinois 60510, USA}

\author{Z. Xi}
\affiliation{Department of Engineering and Physics, Abilene Christian
University, Abilene, Texas 79699, USA}

\author{Z. Ye}
\affiliation{Physics Division, Argonne National Laboratory, Lemont,
Illinois 60439, USA}

\collaboration{FNAL E906/SeaQuest Collaboration}
\noaffiliation

\date{\today}
\begin{abstract}
High-mass dimuon production in $p+p$ and $p+d$ interaction has been measured
in the SeaQuest experiment with 120 GeV proton beam at Fermilab.
We report the $(p+d) / (p+p)$ cross section
ratios for $J/\Psi$ and $\psi^\prime$ production covering the forward
rapidity region of $0.3 < x_F <0.8$. Unlike the recently reported
$(p+d) / (p+p)$ Drell-Yan cross section ratios from SeaQuest, which are
sensitive to the
$\bar d / \bar u$ antiquark distributions in the proton, the corresponding
cross section ratios for charmonium production are primarily
sensitive to the gluon
distributions in the proton and neutron. We compare the measured
charmonium production cross section ratios with that of the
Drell-Yan process. We also compare the measured cross section ratios with
theoretical calculations.

\todo{include psip as well}
\end{abstract}

\pacs{14.20.Dh,14.65.Bt,13.60.Hb}

\maketitle
\section{Introduction}
\todo{explain the interest of studying the charmomnium and NRQCD}
\todo{explain the scope of the paper}

\section{The SeaQuest Experiment}
\todo{brief discussion of the spectrometer}

\section{Data Analysis}
\todo{explain mass fit, refer to long paper}
\begin{figure}
\missingfigure{mass fit plot}
\caption{Dimuon mass distribution }
\label{fig:massfit}
\end{figure}

\section{$J/\psi$ and $\psi'$ Cross section}
\todo{describe the cross section calculation}
\todo{describe the extracted cross section and how it compared with predictions}
\begin{figure}
\missingfigure{cross section vs $x_F$}
\caption{The measured $\dv{\sigma}{x_F}$ for $J/\psi$ and $\psi'$ compared with NRQCD calculations}
\label{fig:xF_cross_sections}
\end{figure}

\todo{The measured $\expval{P_T^2}$ vs $\sqrt{s}$ from different experiments}
\begin{figure}
\missingfigure{$\expval{P_T^2}$ vs $\sqrt{s}$}
\caption{The measured $\expval{P_T^2}$ from SeaQuest for $p+p\rightarrow J/\psi$ compared with other experiments }
\label{fig:pt_s}
\end{figure}

\nocite{*}
\bibliography{reference}
\end{document}
