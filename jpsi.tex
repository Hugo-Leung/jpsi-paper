\documentclass[twocolumn,aps,unsortedaddress,superscriptaddress,prd,floatfix,showpacs,linenumbers]{revtex4-2}
\usepackage[utf8]{inputenc}
\usepackage{float}
\usepackage{graphicx}
\usepackage{amsmath,amsthm,amssymb}
\usepackage{verbatim}
\usepackage{subcaption}
\usepackage[separate-uncertainty=true]{siunitx}
\usepackage{physics}
\usepackage{cleveref}
\usepackage[obeyFinal]{todonotes}
\setuptodonotes{inline}
\graphicspath{{figures/}}

\begin{document}
\title{Measurement of $(p+d) / (p+p)$ cross section ratios for
$J/\Psi$ and $\Psi^\prime$ production at \SI{120}{\GeV}}
\affiliation{Abilene Christian University, Abilene, Texas 79699, USA}
\affiliation{Academia Sinica, Taipei, 11529, Taiwan}
\affiliation{Argonne National Laboratory, Lemont, Illinois 60439, USA}
\affiliation{Fermi National Accelerator Laboratory, Batavia, Illinois 60510, USA}
\affiliation{Los Alamos National Laboratory, Los Alamos, New Mexico 97545, USA}
\affiliation{KEK, High Energy Accelerator Research Organization, Tsukuba, Ibaraki 305-0801, Japan}
\affiliation{Mississippi State University, Mississippi State, MS 39762, USA }
\affiliation{National Kaohsiung Normal University, Kaohsiung City 80201, Taiwan}
\affiliation{RIKEN Nishina Center for Accelerator-Based Science, Wako, Saitama 351-0198, Japan}
\affiliation{Rutgers, The State University of New Jersey, Piscataway, New Jersey 08854, USA}
\affiliation{Tokyo Institute of Technology, Meguro-ku,Tokyo 152-8550, Japan}
\affiliation{University of Colorado, Boulder, Colorado 80309, USA} 
\affiliation{University of Illinois at Urbana-Champaign, Urbana, Illinois 61801, USA}
\affiliation{University of Maryland, College Park, Maryland 20742, USA}
\affiliation{University of Michigan, Ann Arbor, Michigan 48109, USA}
\affiliation{Yamagata University, Yamagata City, Yamagata 990-8560, Japan}

\author{C. Leung}
\affiliation{University of Illinois at Urbana-Champaign, Urbana, Illinois 61801, USA}

\author{J. Dove}
\affiliation{University of Illinois at Urbana-Champaign, Urbana, Illinois 61801, USA}

\author{B. Kerns}
\affiliation{University of Illinois at Urbana-Champaign, Urbana, Illinois 61801, USA}

\author{R. E. McClellan}
\affiliation{University of Illinois at Urbana-Champaign, Urbana, Illinois 61801, USA}
%\footnote[1]{Now at Thomas Jefferson National Accelerator Facility, Newports News, IL 23696, USA}}

\author{S. Miyasaka}
\affiliation{Tokyo Institute of Technology, Meguro-ku,Tokyo 152-8550, Japan}

\author{D. H. Morton} 
\affiliation{University of Michigan, Ann Arbor, Michigan 48109, USA}

\author{K. Nagai}
\affiliation{Academia Sinica, Taipei, 11529, Taiwan}
\affiliation{Tokyo Institute of Technology, Meguro-ku,Tokyo 152-8550, Japan}

\author{S. Prasad}
\affiliation{University of Illinois at Urbana-Champaign, Urbana, Illinois 61801, USA}

\author{F. Sanftl}
\affiliation{Tokyo Institute of Technology, Meguro-ku,Tokyo 152-8550, Japan}

\author{M. B. C. Scott}
\affiliation{University of Michigan, Ann Arbor, Michigan 48109, USA}

\author{A. S. Tadepalli}
\affiliation{Rutgers, The State University of New Jersey, Piscataway, New Jersey 08854, USA}

\author{C. A. Aidala}
\affiliation{University of Michigan, Ann Arbor, Michigan 48109, USA}
\affiliation{Los Alamos National Laboratory, Los Alamos, New Mexico 97545, USA}

\author{J.  Arrington}
\affiliation{Argonne National Laboratory, Lemont, Illinois 60439, USA}

\author{C. Ayuso}
\affiliation{University of Michigan, Ann Arbor, Michigan 48109, USA}

%\author{K. G. Bailey$^c$,}

\author{C. L. Barker}
\affiliation{Abilene Christian University, Abilene, Texas 79699, USA}

\author{C. N. Brown}
\affiliation{Fermi National Accelerator Laboratory, Batavia, Illinois 60510, USA}

\author{W.C. Chang}
\affiliation{Academia Sinica, Taipei, 11529, Taiwan}

\author{A. Chen}
\affiliation{University of Illinois at Urbana-Champaign, Urbana, Illinois 61801, USA}
\affiliation{Academia Sinica, Taipei, 11529, Taiwan}
\affiliation{University of Michigan, Ann Arbor, Michigan 48109, USA}

\author{D. C. Christian}  
\affiliation{Fermi National Accelerator Laboratory, Batavia, Illinois 60510, USA}

\author{B. P. Dannowitz}
\affiliation{University of Illinois at Urbana-Champaign, Urbana, Illinois 61801, USA}

\author{M. Daugherity}
\affiliation{Abilene Christian University, Abilene, Texas 79699, USA}

\author{M. Diefenthaler}
\affiliation{University of Illinois at Urbana-Champaign, Urbana, Illinois 61801, USA}

\author{L. El Fassi} 
\affiliation{Mississippi State University, Mississippi State, MS 39762, USA }
\affiliation{Rutgers, The State University of New Jersey, Piscataway, New Jersey 08854, USA}

\author{D. F. Geesaman}
\affiliation{Argonne National Laboratory, Lemont, Illinois 60439, USA}
%\footnote[2]{Now an Argonne Associate of Global Empire LLC, Lemont IL 60439, USA},} 

\author{R. Gilman}
\affiliation{Rutgers, The State University of New Jersey, Piscataway, New Jersey 08854, USA}

\author{Y. Goto}
\affiliation{RIKEN Nishina Center for Accelerator-Based Science, Wako, Saitama 351-0198, Japan}

\author{L. Guo}
\affiliation{Los Alamos National Laboratory, Los Alamos, New Mexico 97545, USA}
%\footnote[3]{now at Florida International University, Miami, Florida, 33199, USA},}

\author{R. Guo}
\author{T. J. Hague}
\affiliation{Abilene Christian University, Abilene, Texas 79699, USA}

\author{R. J. Holt}
\affiliation{Argonne National Laboratory, Lemont, Illinois 60439, USA}
%\footnote[4]{Now at Kellogg Radiation Laboratory, California Institute of Technology, Pasadena, CA, 91125, USA},} 

\author{D. Isenhower} 
\affiliation{Abilene Christian University, Abilene, Texas 79699, USA}

\author{E. R. Kinney}
\affiliation{University of Colorado, Boulder, Colorado 80309, USA} 

\author{N. Kitts}
\affiliation{Abilene Christian University, Abilene, Texas 79699, USA}

\author{A. Klein}
\affiliation{Los Alamos National Laboratory, Los Alamos, New Mexico 97545, USA}

\author{D. W. Kleinjan}
\affiliation{Los Alamos National Laboratory, Los Alamos, New Mexico 97545, USA}

\author{Y. Kudo}
\affiliation{Yamagata University, Yamagata City, Yamagata 990-8560, Japan}

\author{P.-J. Lin}
\affiliation{University of Colorado, Boulder, Colorado 80309, USA} 

\author{K. Liu} 
\affiliation{Los Alamos National Laboratory, Los Alamos, New Mexico 97545, USA}

\author{M. X. Liu} 
\affiliation{Los Alamos National Laboratory, Los Alamos, New Mexico 97545, USA}

\author{W. Lorenzon}
\affiliation{University of Michigan, Ann Arbor, Michigan 48109, USA}

\author{N. C. R. Makins}
\affiliation{University of Illinois at Urbana-Champaign, Urbana, Illinois 61801, USA}

\author{ M. Mesquita de Medeiros}
\affiliation{Argonne National Laboratory, Lemont, Illinois 60439, USA}

\author{P. L. McGaughey}
\affiliation{Los Alamos National Laboratory, Los Alamos, New Mexico 97545, USA}

\author{Y. Miyachi} 
\affiliation{Yamagata University, Yamagata City, Yamagata 990-8560, Japan}

\author{I. Mooney}
\affiliation{University of Michigan, Ann Arbor, Michigan 48109, USA}
%\footnote[5]{Now at Wayne State University, Detroit, MI 48202, USA},}

\author{K. Nakahara}
\affiliation{University of Maryland, College Park, Maryland 20742, USA}
%\footnote[6]{Now at Stanford Linear Accelerator Center, Menlo Park, CA 94025, USA},}

\author{K. Nakano}
\affiliation{Tokyo Institute of Technology, Meguro-ku,Tokyo 152-8550, Japan}
\affiliation{RIKEN Nishina Center for Accelerator-Based Science, Wako, Saitama 351-0198, Japan}

\author{S. Nara} 
\affiliation{Yamagata University, Yamagata City, Yamagata 990-8560, Japan}

\author{J.-C. Peng}
\affiliation{University of Illinois at Urbana-Champaign, Urbana, Illinois 61801, USA}

\author{A. J. Puckett}
\affiliation{Los Alamos National Laboratory, Los Alamos, New Mexico 97545, USA}
%\footnote[7]{Now at University of Connecticut, Storrs, CT 06269, USA},}

\author{B. J. Ramson}
\affiliation{University of Michigan, Ann Arbor, Michigan 48109, USA}
\affiliation{Fermi National Accelerator Laboratory, Batavia, Illinois 60510, USA}
%\footnote[8]{Now at Fermi National Accelerator Laboratory, Batavia IL 60510, USA},} 

\author{P. E. Reimer} 
\affiliation{Argonne National Laboratory, Lemont, Illinois 60439, USA}

\author{J. G. Rubin}
\affiliation{University of Michigan, Ann Arbor, Michigan 48109, USA}
\affiliation{Argonne National Laboratory, Lemont, Illinois 60439, USA}

\author{S. Sawada} 
\affiliation{KEK, High Energy Accelerator Research Organization, Tsukuba, Ibaraki 305-0801, Japan}

\author{T. Sawada}
\affiliation{University of Michigan, Ann Arbor, Michigan 48109, USA}
%\footnote[9]{Now at Osaka City University, Sumiyoshi-ku, Osaka City, Osaka 558-8585, Japan},}

\author{T.-A. Shibata}
\affiliation{Tokyo Institute of Technology, Meguro-ku,Tokyo 152-8550, Japan}
%\footnote[10]{Now at Nihon University, Tokyo 101-8308, Japan},} 

\author{D. Su}
\affiliation{Academia Sinica, Taipei, 11529, Taiwan}

\author{M. Teo}
\affiliation{University of Illinois at Urbana-Champaign, Urbana, Illinois 61801, USA}
%\footnote[11]{Now at Stanford University, Stanford, CA 94305},}

\author{B. G Tice}
\affiliation{Argonne National Laboratory, Lemont, Illinois 60439, USA}

\author{R. S. Towell}
\affiliation{Abilene Christian University, Abilene, Texas 79699, USA}

\author{S. Uemura}
%\footnote[12]{Now at Tel Aviv University, Tel Aviv 69978, Israel},}
\author{S. Watson}
\affiliation{Abilene Christian University, Abilene, Texas 79699, USA}

\author{S. G. Wang}
\affiliation{Academia Sinica, Taipei, 11529, Taiwan}
\affiliation{National Kaohsiung Normal University, Kaohsiung City 80201, Taiwan}
%\footnote[13]{Now at The University of Chicago, Lemont, IL, 60439 USA},}

\author{A. B. Wickes}
\affiliation{Los Alamos National Laboratory, Los Alamos, New Mexico 97545, USA}

\author{J. Wu}
\affiliation{Fermi National Accelerator Laboratory, Batavia, Illinois 60510, USA}

\author{Z. Xi}
\author{Z. Ye} 
\affiliation{Argonne National Laboratory, Lemont, Illinois 60439, USA}

\collaboration{FNAL E906/SeaQuest Collaboration}
\noaffiliation


\date{\today}
\begin{abstract}
High-mass dimuon production in $p+p$ and $p+d$ interaction has been measured
in the SeaQuest experiment with 120 GeV proton beam at Fermilab.
We report the $(p+d) / (p+p)$ cross section
ratios for $J/\Psi$ and $\psi^\prime$ production covering the forward
rapidity region of $0.3 < x_F <0.8$. Unlike the recently reported
$(p+d) / (p+p)$ Drell-Yan cross section ratios from SeaQuest, which are
sensitive to the
$\bar d / \bar u$ antiquark distributions in the proton, the corresponding
cross section ratios for charmonium production are primarily
sensitive to the gluon
distributions in the proton and neutron. We compare the measured
charmonium production cross section ratios with that of the
Drell-Yan process. We also compare the measured cross section ratios with
theoretical calculations.

\todo{include psip as well}
\end{abstract}

\pacs{14.20.Dh,14.65.Bt,13.60.Hb}

\maketitle
\section{Introduction}
\todo{explain the interest of studying the charmomnium and NRQCD}
\todo{explain the scope of the paper}

\section{The SeaQuest Experiment}
\todo{brief discussion of the spectrometer}

\section{Data Analysis}
\todo{explain mass fit, refer to long paper}
\begin{figure}
\missingfigure{mass fit plot}
\caption{Dimuon mass distribution }
\label{fig:massfit}
\end{figure}

\section{$J/\psi$ and $\psi'$ Cross section}
\todo{describe the cross section calculation}
\todo{describe the extracted cross section and how it compared with predictions}
\begin{figure}
\missingfigure{cross section vs $x_F$}
\caption{The measured $\dv{\sigma}{x_F}$ for $J/\psi$ and $\psi'$ compared with NRQCD calculations}
\label{fig:xF_cross_sections}
\end{figure}

\todo{The measured $\expval{P_T^2}$ vs $\sqrt{s}$ from different experiments}
\begin{figure}
\missingfigure{$\expval{P_T^2}$ vs $\sqrt{s}$}
\caption{The measured $\expval{P_T^2}$ from SeaQuest for $p+p\rightarrow J/\psi$ compared with other experiments }
\label{fig:pt_s}
\end{figure}

\nocite{*}
\bibliography{reference}
\end{document}
